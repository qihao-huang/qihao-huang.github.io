\resheading{工作经历}
  % 把一个东西讲得简单
  
  % * 以下全职实习经历总计约两年。
  
  \begin{itemize}[leftmargin=*]
    \item
      \ressubsingleline{\href{https://developer.aliyun.com/group/ailab}{阿里巴巴 (杭州) 阿里云 - AI Lab}}{计算机视觉算法工程师}{2021.11 -- 现在}
      % {\small
      % \begin{itemize}
      % \begin{spacing}{1.2}
      %   \item 岗前实习。
      % \end{spacing}
      % \end{itemize}
      % }
  
  % \vspace{-1.5em}

    \item
      \ressubsingleline{\href{https://www.apple.com}{苹果 (北京) - Video Engineering}}{2D \& 3D 计算机视觉算法实习生}{2021.05 -- 2021.10}
      {\small
      \begin{itemize}
      \begin{spacing}{1.2}
        \item 暑期实习。
        \item 受限于保密政策。
        
      \end{spacing}
      \end{itemize}
      }
  
  \vspace{-1.5em}
  
    \item
      \ressubsingleline{\href{https://ailab.bytedance.com/}{字节跳动 (北京) Data - 商业化/AI Lab}}{视觉计算组算法实习生}{2021.01 -- 2021.05}
      {\small
      \begin{itemize}
      \begin{spacing}{1.2}
        \item 与\textbf{\href{https://scholar.google.com/citations?user=iHoGTt4AAAAJ}{佘琪博士}}及 Ads Core, 穿山甲团队合作,
        尝试从在线学习 (Online Learning) 优化器的角度解决现有广告 CVR 模型中存在的细粒度周期性衰减问题,即模型快速适应人群分布的时序变换,却遗忘了过去的历史信息和偏好,造成广告点击转化率下降和广告主价值损失。
    
        \item 在 Parameter Server 中设计、开发 O2GD, FTML, MADGrad 等开源优化方案。 设计稀疏优化 FTRML 及 AdaMom 优化器,后者在穿山甲业务中取得离线万十的 AUC 收益,并持续推动流式训练中。
        
      \end{spacing}
      \end{itemize}
      }

  \vspace{-1.5em}
  
    \item
      \ressubsingleline{\href{https://www.xyzrobotics.ai/?page_id=12903}{XYZ Robotics (上海)}}{计算机视觉算法核心研发实习生}{2019.01 -- 2019.08}
      {\small
      \begin{itemize}
      \begin{spacing}{1.2}
        \item 与\textbf{\href{http://www.cs.cmu.edu/~jiajiz/}{周佳骥博士}、\href{http://people.csail.mit.edu/peterkty/}{俞冠廷博士}}合作,研发以视觉算法驱动的物流自动化解决方案,
          应用于\href{https://www.xyzrobotics.ai/?page_id=12903}{\textbf{小件拣选}和\textbf{整箱拆垛}}项目。
    
        \item \textbf{项目一:紧密堆放物品}场景下的小件拣选研发,在团队于 \href{https://arc.cs.princeton.edu/}{Amazon Robotics Challenge MIT-Princeton Winning Solution} 
          混杂场景的基础上,扩展至紧密堆放场景。为解决紧密堆放时吸取点预测无法准确定位到物体中心的问题,我们以边缘预测作为约束。
          此外,我们通过 RGB-DDD 融合输入的方式,提高由相机摆放导致的远侧物体大小不一致下的预测成功率。
          在欧莱雅、天猫合作提供的30余种不同颜色、纹理、角度、大小的物体、多种堆叠方式中实现100\%的密集识别测试成功率。此项目获得团队\href{https://qihao-huang.github.io/archive/HQH-XYZ-intern-award.pdf}{\textbf{年度实习生}}的额外奖励。
          [\href{https://www.xyzrobotics.ai/?page_id=12903}{详情链接}]
        
        \item \textbf{项目二:整箱拆垛}的新项目研发,将垛箱俯视图识别转换为实例分割问题,扩展 Mask R-CNN 作为视觉核心。由于新业务的特殊性,于某市医药企业自动化地采集仓库搬运数据。
          考虑到采集到的数据规模较小,使用 Blender 合成不同纹理、大小、随机噪声的垛箱俯视图以增加多样性。在此基础之上,此项目于2020年成为公司核心解决方案。
        
        % 渲染引擎 Python 接口
        % 在本地服务器上编写 Labelme 公网端口,众包数据团队标注GT。
        % 与硬件团队合作,在企业内部 ABB 机器人平台上部署。过程中,学习使用 RealSense RGB-D 相机的 PCL C++ 接口,棋盘标定等。
               
      \end{spacing}
      \end{itemize}
      }
  
  \vspace{-1.5em}
  \end{itemize}