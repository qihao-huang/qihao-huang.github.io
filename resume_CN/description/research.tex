\resheading{研究经历}
  \begin{itemize}[leftmargin=*]
    \item
      \ressubsingleline{从单目视频中恢复场景深度并学习动态物体的3D运动}{港大硕士毕业论文课题}{2020.02 -- 2021.02}
      {\small
      \begin{itemize}
      \begin{spacing}{1.2}
        \item 在\textbf{\href{https://xjqi.github.io/} {齐晓娟教授}}的指导下,以共同一作提交\href{https://qihao-huang.github.io/archive/DO3D_iros2021_manuscript.pdf}
          {``Self-supervised Learning of Decomposed Object-wise 3D Motion and Depth from Monocular Videos"}至 IROS 2021.
          [\href{https://qihao-huang.github.io/archive/DO3D_iros2021_manuscript.pdf}{PDF链接}]
        \item 本课题旨在建立自监督方式下的动态物体运动预测和深度估计框架。以往工作的自监督信号仅来源于从邻接帧学习到的相机姿态和场景深度,
          未考虑到\textbf{动态物体}的影响,因此无法全局优化。我们提出了\textbf{多阶段预测刚性物体和非刚性物体的运动模型}来解决这一问题,从而\textbf{分解复杂3D运动},以此纠正并提高光流预测和单目深度估计性能。
        \item 我们的算法在单目自监督下的动态前景物体光流预测性能较 baseline (GeoNet) 提升了22.8\%. 
      \end{spacing}
      \end{itemize}
      }
    \vspace{-1.5em}

    % \item
    %   \ressubsingleline{\href{https://eval.ai/web/challenges/challenge-page/829/overview}{Supervised Learning Track -- ``Synbols''}}
    %     {\href{https://sites.google.com/view/clvision2021}{\rm{CVPR 2021 Workshop Challenge}}}{2021.03 -- 2021.05}
    %   {\small
    %   \begin{itemize}
    %   \begin{spacing}{1.2}
    %     \item 在\href{https://scholar.google.com/citations?user=iHoGTt4AAAAJ}{佘琪博士}的指导下,参与 CVPR 2021 持续学习专题 ($2^{nd}$ Continual Learning Workshop) 竞赛 Synbols 环节。
    %     \item 我们设计并采用 Online Orthogonal Gradient Descent (O2GD) 的方法来克服 Synbols Incremental Image Classification Tasks 中面临的灾难性遗忘问题 (Catastrophic Forgetting).(正在实验中)

    %   \end{spacing}
    %   \end{itemize}
    %   }
    % \vspace{-1.5em}

  \end{itemize}