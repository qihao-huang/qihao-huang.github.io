% Situation:事情发生的情境;
% Task:你的任务是什么;
% Action:你的行动是什么;
% Result:结果如何,你取得了什么成果。

\resheading{项目经历}
  \begin{itemize}[leftmargin=*]
     \item
      \ressubsingleline{\href{https://github.com/qihao-huang/app-cv}{xxx}}{yyy}{2020.04 -- 2020.06}
      {\small
      \begin{itemize}
        \begin{spacing}{1.2}
        \item xxx 
      \end{spacing}
      \vspace{0.1em}
      \end{itemize}
      }
      
    %  \item
    %   \ressubsingleline{\href{https://github.com/qihao-huang/cpp-python-socket}{C++ 和 Python 间的通讯接口,辅助视觉算法实时测试}}{个人开发}{2019.05 -- 2019.06}
    %   {\small
    %   \begin{itemize}
    %   \begin{spacing}{1.2}
    %     \item 此项目为实习中的小工具。为频繁测试新模型的实际表现并解耦机器人模块,将相机 C++ API 得到的 RGB-D 图片发送 PyTorch 端并实时显示预测效果,以此矫正相机位姿和调整模型,此工具支持多通道 RGB 和 16 位深度图。
    %   \end{spacing}
    %   \vspace{0.1em}
    %   \end{itemize}
    %   }
      
    %  \item
    %   \ressubsingleline{\href{https://qihao-huang.github.io/monocular-depth-prediction/presentation.pdf}{基于卷积神经网络的多尺度深度估计}}{浙大计算摄影学课程项目}{2018.04 -- 2018.06}
    %   {\small
    %   \begin{itemize}
    %     \item 此项目以 PyTorch 复现 2014/2015 Eigen NYU Depth-V2 论文代码,接入 ResNet 混合训练。
    %   \end{itemize}
    %   }
  \end{itemize}