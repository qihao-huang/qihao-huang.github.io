\resheading{Research Exprience}
  \begin{itemize}[leftmargin=*]
    \item
      \ressubsingleline{Learn 3D motion and depth of dynamic objects from monocular videos}{}{2020.02 -- 2021.02}
      {\small
      \begin{itemize}
      \begin{spacing}{1.2}
        \item Master thesis, advised by \textbf{\href{https://xjqi.github.io/} {Prof. Xiaojuan Qi}}, submitted  
          \href{https://qihao-huang.github.io/archive/DO3D_iros2021_manuscript.pdf}{\textbf{``Self-supervised Learning of Decomposed Object-aware 3D Motion and Depth from Monocular Videos"}} to IROS 2021 as the co-first author. 
          \href{https://qihao-huang.github.io/archive/DO3D_iros2021_manuscript.pdf}{[PDF]}
        \item This project aims to establish a framework for learning 3D motion of dynamic objects 
          and scene depth in a self-supervised manner. The self-supervision signal of previous works come from 
          camera pose and scene depth of consecutive frames without considering the influence of those dynamic objects. 
          It composes the incapability of global optimization. Hence, we propose a \textbf{multi-stage decomposition model to predict rigid and non-rigid motion}
           to adjust the foreground optical flow (ego and object-wise motion) and depth prediction.
        \item Our algorithms boost the baseline (GeoNet, Monodepth2) optical flow performance in foreground dynamic areas 
        by 22.8\% in the field of self-supervised monocular manner.
      \end{spacing}
      \end{itemize}
      }
  \vspace{-1.5em}
  \end{itemize}